
This section will explain the development methodology, technologies and tools we used to develop the project. 

\subsection{Methodology}
%argumenter hvorfor for vi har valgt det.
%Tilpassinger, prsaker til tilpassinger.
%Vi har valgt kamban 
%Fordi


This project has been developed using the Kanban-methodology which is a lean (and agile) development methodology. In a nutshell, the Kanban-methodology is about three principles \citep{Kniberg2010}: 
\begin{itemize}
\item Visualize work flow 
\item Limit work in progress 
\item Measure the lead time
\end{itemize}

The most important aspects for our project was the flexibility this method gives. Being an agile method we could develop the project incrementally. This is good because our goal was to develop a working high fidelity prototype. We could also take advantage of the fact that it doesn't prescribe meetings, roles and practices the way Scrum or XP does \citep{Abrahamsson2002}. We (the team) could freely decide what to do, how to do it and when to do it. Because there is no rules in Kanban about what tasks to do next and it's allowed to add and remove task during the whole development. This was good as we were unsure about the "right" way to retrieve information from Twitter. The principle of measuring lead time is only important in a business situation and it didn't apply to this project.  

In the "kickoff" of the project we made a backlog of User Stories. Through the development we split the stories into smaller development tasks. Each team member chose development tasks and posted them on the kanban board. The board also helped us keep track of what the others were doing. It also helped us limit the work in progress because we would finish a task before starting a new one. We used an electronic kanban board from Kanbanpad.com \citep{TheHybridGroup2012} to help us visualize the work flow. Kanbanpad is a free and online Kanban board with a simple interface and support for multiple platforms \citep{TheHybridGroup2012}. Screen shots of the kanban board from two states in the development process can be found in the attachments figure \ref{fig:kanbanScreenShot} and \ref{fig:kanbanScreenShot2}. An example of how we defined the development tasks can be found in attachments figure \ref{fig:kanbanScreenShotDevTask}. 


\subsection{Tools}
During the project life cycle a number of tools have been used for the development process, implementation of code and deployment of the web application itself. A brief description of the tools used will be provided in this section.

\subsubsection{Implementation}
To implement the Java and Spring Framework based backend the SpringSource Tool Suite (figure~\ref{fig:springsourcetoolsuite} was used. \textit{``SpringSource Tool Suite™ (STS) provides the best Eclipse-powered development environment for building Spring-powered enterprise applications.''} \cite{SpringSource}. SpringSource Tool Suite is based on the open-source IDE framework Eclipse and provides functionality such as close integration with the Spring Framework, maven integration, automatic application deployment to web servers, debugging, code-completion and much more.

The front-end code was implemented using mainly Notepad++ (figure~\ref{fig:notepadplusplus}) for coding and Google Chrome development tools for debugging and testing. \textit{``Notepad++ is a free (as in "free speech" and also as in "free beer") source code editor and Notepad replacement that supports several languages. Running in the MS Windows environment, its use is governed by GPL License.''} \cite{Ho2012}.

All the code was, when ready, shared with the rest of the team through a common online repository. The repository host used was GitHub.com. \textit{``GitHub is the best place to share code with friends, co-workers, classmates, and complete strangers. Over a million people use GitHub to build amazing things together.''} \cite{GitHub}. GitHub utilises the widely used source code revision and management system Git. \textit{``Git is a free and open source distributed version control system designed to handle everything from small to very large projects with speed and efficiency.''} \cite{Git}. Through using Git and GitHub we were able to work on any files at any time and simply merge any conflicting changes to any classes. We were also able to roll back any unwanted changes, due to GitHub storing a changelist.

\subsubsection{Deployment}
The application is deployed on a server running the Linux distribution Ubuntu 12.04, a Tomcat web server, a LAMP-stack (Linux, Apache, MySQL, PHP) and Maven. Ubuntu 12.04 is the newest version of the operative system developed by Canonical and is suitable for use as a web server with desktop access. To serve the webpages of the FeedJam application we use Tomcat, an open-source web server for Java-based web applications. A MySQL server is used to drive the application's database of users and tweets. An Apache php server is used to provide the phpMyAdmin web interface to manage the MySQL server.

\subsection{Technology}
This section gives a short presentation of the technologies we used in the project. 

\subsubsection{HTTP Requests}
In order to send data from a browser to a server one utilises the HTTP request methods. There are two different types of HTTP requests: HTTP GET and HTTP POST. The main differences between these two methods is that data submitted through a GET is encoded by the browser and appended to the URL while data submitted through POST is added to the HTTP request message body.

The practical difference is that POST can take larger amounts of data than GET and that GET requests can be made easily through a browser through manipulating URLs. Therefore POST is mainly used to send data, while GET is mainly used to request data.

\subsubsection{Javascript/jQuery} %torstein
This project relies heavily on client-side Twitter requests in order to provide each user with enough requests to make the application usable. Javascript is a scripting language, at the moment the only one, which is run client-side in browsers. While Javascript itself is quite straight forward, following the ECMAscript standard, there are several inconsistencies in its implementation in different browsers. JavaScript thus demands large amounts of experience in order to develop stable and functioning scripts. jQuery is a library written in Javascript which tries to account for these inconsistencies in addition to providing easier syntax and a large amount of prepared methods.

\subsubsection{jQuery Masonry}
Masonry is an open source jQuery plugin which enables better placement of elements within a html grid. It works through calculated the rendered size of each element in the grid, and the placing each element with absolute pixel values in the final grid. It is used in the layout of FeedJam to provide a smoother display of the tweets.

\subsubsection{jQuery Revolver}
Revolver is an open source jQuery plugin which lets us generate a slider from a set of HTML elements. In the context of FeedJam jQuery Revolver is used to generate the sliding effect on the trends displayed on the front page.

\subsubsection{Media Queries} %torstein
In order to customise the layout of websites for different screen resolutions one can employ the use of CSS media queries. Media queries makes it possible to determine what type of media a web page is accessed from with pure CSS code. While being a quite new innovation, media queries is supported by most browsers, including webkit and fennec based browsers, which includes most browsers used on smart phones and tablets. Media queries follow a quite simple syntax, and can be used within the standard css-document. For instance CSS written within "@media only screen and (min-width: 768px) and (max-width: 991px) \{" and "\}" would only be used in cases where the browser window has a width of between 768 and 991 pixels. Media queries are used in FeedJam in order to make the layout more available to smart phone and tablet users.

\subsubsection{AJAX}%torstein
AJAX, standing for Asynchronous Javascript and XML, is a technique where one can enable client-server interaction after a page has been rendered and send to the user's browser. In the context of our project we use the equivalent technique, AJAJ, which uses JSON instead of XML. On an abstract level one can explain AJAX as server requests which does not cause a page load, and whose response can be used either to insert new content dynamically or for some other purpose by the client. FeedJam uses this to, among other things, cache followers and following after the view with the search result is returned. 

\subsubsection{JSON}
JSON (JavaScript Object Notation) is a data-interchange format \cite{Crockford2011}. JSON has several benefits when compared to other data-interchange formats such as XML. For instance, since it is based on Javascript object and array notation, it can be directly imported into Javascript. It is also lighter weight and arguably easier to write and read than the equivalent XML. In our project we receive data from the Twitter APIs in a JSON format.

\subsubsection{Spring} %lisa
To develop the project we used the Spring Framework. It is a Java based open source application framework. We chose this framework because it has many modules that easily takes care of the "plumbing" of the application and makes our work easier \citep{SpringSourcea}. Some of the group members has used it before, and all wanted to get to know it better because it one of the big enterprise frameworks. The application is set up using Spring MVC architecture which allows for easy communication between the model, the web view and the controller (more on this in section \ref{sec:MVC} \nameref{sec:MVC}) \citep{SpringSourced}. FeedJam uses Spring's REST module for client-side HTTP access to Twitter's API. We also make user of Spring's data access module for connection to the database.
