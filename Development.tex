\subsection{Tools} %snorre
During the project life cycle a number of tools have been used for the development process, implementation of code and deployment of the web application itself. A brief description of the tools used will be provided in this section.

\subsubsection{Implementation}
To implement the Java and Spring Framework based backend the SpringSource Tool Suite was used. \textit{``SpringSource Tool Suite™ (STS) provides the best Eclipse-powered development environment for building Spring-powered enterprise applications.''} \cite{SpringSource}. SpringSource Tool Suite is built on the open-source IDE framework Eclipse, and provides functionality such as close integration with the Spring Framework, maven integration, automatic application deployment to web servers, debugging, code-completion and much more.

The front-end code was implemented using mainly Notepad++ for coding and Google Chrome development tools for debugging and testing. \textit{``Notepad++ is a free (as in "free speech" and also as in "free beer") source code editor and Notepad replacement that supports several languages. Running in the MS Windows environment, its use is governed by GPL License.''} \cite{Ho2012}.

\subsubsection{Deployment}
The application is deployed on a server running the Linux-distro Ubuntu 12.04, a Tomcat web server, a LAMP-stack (Linux, Apache, MySQL, PHP) and Maven. Ubuntu 12.04 is the newest version of the operative system developed by Canonical and is suitable for use as a web server with desktop access. To serve the webpages of the FeedJam application we use Tomcat. 

%server
\subsection{Workflow}
\subsection{Methodology}
%argumenter hvorfor for vi har valgt det.
%Tilpassinger, prsaker til tilpassinger.

This project has been developed using the Kanban-methodology which is a lean (and agile) development methodology


This project has been developed using a Kanban. Kanban is a Lean software development methodology with five core principles:
Vizualize work flow 
Limit work in progress 
Manage flow
Make process policies explicite
Improve collaboration

At the start of the project we made a backlog of User Stories. We used an electronic kanban board from Kanbanpad.com to visualize the work flow. Through the development we split the stories into smaller development tasks and posted on the kanban board. This helped us limit the work in progress because we would finish a task before starting a new one. (See attachments for screenshots from the kanban board). 

Technologies

Javascript/jQuery
HTML 5
CSS 3
Media Queries
AJAX
Git
Java
Spring
Jetty/Tomcat
MySQL
Workflow

Git/GitHub
Kanban