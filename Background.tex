%teoridel
In this section we will present some background information about the topics that the project consist of. We'll star by introducing ranking in relation to search engines. After this we will take a closer look at the algorithm that we will use in our project, the PageRank algorithm.  Following will be a short introduction to Twitter, the social media that is the domain of the project. The section will finish with a presentation of some related work and a subsection of your motivation for doing the project. The next section .....


\subsection{Ranking} 
In defining the goals for this project we chose to focus on two parts of the web information retrieval process, ranking and presentation of search results. Ranking is the hardest and most important function of a search engine \citep[p.469]{Baeza-Yates2011}. One of the key challenges in ranking, according to \citet{Baeza-Yates2011}, is to identify quality content on the web. Evidence of quality can be indicated by several signals such as domain names [...], text content, and various counts (such as the number of word occurrences), links (like PageRank), and Web page access patterns as monitored by the search engine” \citep[p.468]{Baeza-Yates2011}. 
\citet{Baeza-Yates2011} go on to describe different types of signals that can be used to improve ranking of web results. These will be briefly described here. Content signals refers to the content of the document, mainly text. The signal in the content type refers to things such as word counts, vector space model-based scores and so forth. Structural signals, according to \citet{Baeza-Yates2011}, includes signals such as the links on a page, the text in anchor elements, and in and out-links to a page.  The last signal type describes is the web usage signals. User clicks is one often used signal where the relevance of a result is affected by the number of clicks it receives. Other signals mentioned are geographical location, language, technological context and temporal context (that is web usage over time gathered through cookies or user profiles)\citep{Baeza-Yates2011}.

Content signals will be used in FeedJam's TweetRank. We will explain in more detail about how we use hashtags, mentions and retweets to create a score in section \ref{} \nameref{}

Link-based ranking is based on using structural signals (i.e. counting the number of outgoing and incoming links on a page) to rank a page or document. One ranking algorithm based on this  principle is the Google PageRank algorithm \citep{Page1999}. We will use this to create a UserRank for Twitter users. 

\subsubsection{Page Rank}
%something about that pagerank comes from the idea of citations counting in journals. If a journal is cited in many articles it is most likely an important article. PageRank takes this idea one step forther. 
The goal of the PageRank ranking algorithm is stated as follows: “This ranking called PageRank helps search engines and users quickly make sense of the vast heterogeneity of the World Wide Web” \citep[p. 1]{Page1999}. It derives from citation counting which is a way tell the importance of an article. An article that is cited by many other articles is most likely an important one. 

The PageRank algorithm exploits the structure of the web in the calculation of the rank. The World Wide Web can thus be viewed as a graph with directed edges between nodes. The edges are either links pointing to the page - we call them back-links. Or they are links pointing to another page - called outgoing-links or forward-links. Looking at the structure of the web and trying to create a ranking of web pages, \citet{Page1999}%Page and ... \citet{} 
took the idea of citations counting one step further. In addition to the number of web pages that point to a page, it also looks at the actual pages that link to it. 
\citet{Page1999} noted that giving high ranks just by looking at the number of back-links (incoming links) is problematic because a site might get many back-links from various web pages that are not very important. They provide a solution to this by proposing that not only the number of back-links, but also the rank of the web pages for these back-links can be used when ranking a web page. Given this solution, as shown by \citet{Page1999}, a page receives a high page rank if the sum of the sum of the page ranks of the back-links is high.

The PageRank algorithm simulates a user navigating through the web randomly. The user is currently at page a. From here she randomly clicks on one of the links on page a to get to the next page. The process is repeated. Using the pattern created by many numbers of clicks by the user, it is possible to compute the probability of which the user has visited each page. There are some restrictions to the computation. The web has pages without outgoing links and self-links where the user is not able to move on. Another hensyn we have to consider is the probability that the user moves to another page not using the links on the current page. Med disse hensyn PageRank is calcualted as follows.
Let L(p) be the number of outgoing links of page p and let p1 . . . pn be the pages that point to page a. q is the probability that the user jumps to another page and 1-q is the probability that she follows the link.  Then the PageRank of page a is given by the probability P R(a) of finding the user in that page
P R(a) = q/T + (1 − q)
Xn
i=1
P R(pi)
L(pi)

where\newline
T: total number of pages on the Web graph\newline
q: parameter set by the system (typical value is 0.15) \newline
When computing the algorithm we have to consider the problem of pages with no outgoing links. \citet{Baeza-Yates2011} propose two solutions. One is to put q = 1 for thses pages, the other is to leave them out, and compute their rank at the end based on their parents rank. 


\subsubsection{User Interface}

{\bf Interaction design principles and user-centered design}\newline
Within the field of interaction design we have several different design guidelines and principles for creating usable artefacts. \citet{Sharp2007} name visibiity, feedback, constraints, consistency and affordance as the essential design principles.

\begin{description}
  \item[Visibility] bob kåre
  \item[Feedback]  :D
  \item[Constraints] :D
  \item[Consistency] :D
  \item[Affordance] :D
\end{description}

In addition to exploring ranking algorithms we are going to develop a novel interface to display tweets and their rankings. In the book Interaction Design: Beyond Human-Computer Interaction Sharp et al (2007) present several factors who are paramount to good interaction design. In order to design good interaction one should take into account the usability principles as presented in the book. These are visibility (whether the functions/input mechanisms are visible or hidden), feedback (the interface should show that the system has received and used the input given), constraints (avoiding facilitating invalid user input for given contexts), consistency (similar operations use the same kinds of elements) and affordance (whether users can use familiarisation from experience in order to naturally understand what the interface element does). In addition to this the book also provides a set of metrics which can be utilised to measure the usability of an artefact.\newline

{\bf \noindent Classic search engine result pages}\newline
\citet[p.480]{Baeza-Yates2011} argues that existing search engines are heavily influenced by what they call a extreme simplicity rule. Basically this rule states that search engine user experience must be simplistic enough in nature that users understand it without any previous knowledge, or else they will simply switch to another search engine. This has lead to a very prominent interface paradigm where all the most prominent search engines use the classic search engine result page (SERP) \citep{Baeza-Yates2011}. 
As search engines have started to search several different content types the need to display several different media types in one unified result page have also emerged. This has resulted in two innovations: the “onebox” \citep{Baeza-Yates2011}, which displays specific information from one knowledge domain (e.g weather reports) and universal search results where one combines several different searches into one search result page.


\subsection{Twitter}

\label{Twitter}
%litt om hvordan twitter kan være vanskelig å få oversikt over en strøm. finn statistikk om megden tweets i døgnet.

%vi kan tilby noe mer enn twitter search enginen gir. Vi tilbyr en mer oversiktlig tjeneste. 
%se om det finnes noe som sier noe om hvordna brukere bruekr twitter. 
When deciding what information retrieval (IR) techniques we could apply to Twitter we found that when searching for Tweets it not easy to see what tweets are important and which are not. If you search for at certain topic on Twitter you get at feed of all tweets that match the search query. It can be some users personal diary, a conversation between users, information sharing or spam. Following we wanted to use IR techniques to make it easier to see what Tweets are important. 

Twitter is a microblogging service available for more than 20 languages \citep{Twitter2012}. User post messages, called tweets, about what they find interesting which are read by other users. As Twitter has become popular it has received a broad user group. The difference between Twitter and other social networks is the way  that it defines the relationships in the network. On Twitters a user is following other user and is being followed by other users. The difference is that the relationship doesn't have to be reciprocal. Being a follower means you receive all the tweets of the users you follow. \citep{Kwak2010} finds that 77.9\% of the followers relationships on Twitter are non reciprocal. This is one of the features we will take advantage of in order to rank tweets. 

\citep{boka kap11} Noe om distribusjonen av sider og linker på web i kap 11. .. noe om zipf lov. Pareto distribution similar to the power law. Many small documents Det samme gjelder for Twitter?

%Dette var noe jeg fant i en artikkel. OG det er ikke så vanskelig å implementer noe som sier om en bruker er venn, deler eller søker. Kanskje vi skal gjøre det?
Users use Twitter in different ways. \citet{Akshay2007} finds three man user categories: Information source, Friend and Information Seeker. Information sources are user that updates their status with regular intervals and has many followers. The number of followers is due to the value of the information they give through the Tweets. The Information Seeker is a user who rarely post Tweets, but follows other users on a regular basis. A Friend is a users with many reciprocal relationships. We may use these user categories in the UserRank. 

Twitter also has other features that can be used in ranking. Important words in the text of a tweet can be marked with a hash tags ("\#" ). Tweets can be retweeted which is a mechanism that empowers the spreading of the tweet. \citep{Kwak2010} finds that a retweeted tweet reaches an average of 1000 users no matter what the number of followers the author of the original tweet has. 

We used the Twitter REST API to retrieve tweets and user information. More on how we used it can be found in section \ref{} \nameref{} 

%Lisa: flytta technologies delen til development delen. Føler det hører mer hjemme der.



\subsection{Related Work}
{\bf trst.me} \newline
trst.me \citet{Infochimps2012b} is tool for measuring a Twitter user's reputation. The measure is two dimensional and consists of the Trstrank and the Trstquitient. The Trstrank is a implementation of Google's page rank \citep{Infochimps2012a}, the same rank the FeedJam uses for the userRank. The Trstrank gives a score between 0 and 10 which gives many users the same rank. To differentiate between the users trst.me uses the Trstquotient. It is "an integer between 0 and 100 that quantifies the relationship between a user's Trstrank and their followers count"\citep{Infochimps2012}. The quotient is used to separate spammers from quality users. 

trst.me compared to FeedJam is just a service for finding a Twitter user's trustworthiness. It allows users to enter the screen name of a Twitter user and returns their rank along with other related metrics.  

\subsection{Motivation} %lisa
%personlig: sociale media er i winden. det vil være inteeressant å  gjøre noe med ir for sosiale medier.
The personal motivation for doing this assignment is that we got the chance to acquire new programming skills. We feel it is easier and more inspiring to learn about techniques in information retrieval field when we can use the theories in a practical sense. Twitter was chosen as a domain because social media and microbloging is a poplar research field. And we wanted to explore their API and see if we could make some interesting and enhance the application using theories from the information retrieval field. 

%faglig: lære mer om hvordan søkemotorer ranker ting, presentasjon. Lære med om verktøy som brukes. Er det et behov for å gjøre dette?
Social media are growing in popularity...

