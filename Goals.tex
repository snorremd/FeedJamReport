Our main goals for FeedJam is to explore a novel way of using ranking algorithms, namely PageRank, and novel layouts in order to ease reading of large collections of Tweets. These goals can be split into several components:

\begin{description}
  \item[Twitter API:] Connect to and use the Twitter API.
  \item[Ranking algorithm:] Implement a ranking algorithm, for instance PageRank.
  \item[Novel presentation:] Create and implement a novel way of displaying tweets.
\end{description}

In addition to the main goals there are several things which we wish to explore. These are our secondary goals:

\begin{description}
  \item[PageRank:] We wish to explore if the PageRank algorithm works within the context of Twitter.
  \item[Spring:] We wish to create a web application with the Spring framework.
  \item[AJAX in Spring:] Explore the possibility of using AJAX combined with the Spring framework.
  \item[Caching:] Implement caching for applicable items from the Twitter API.
  \item[User evaluation:] Evaluate the resulting web application.
\end{description}

\subsection{Motivation} %lisa
%personlig: sociale media er i winden. det vil være inteeressant å  gjøre noe med ir for sosiale medier.
Our personal motivation for making FeedJam was that we got the change to acquire new programming skills. We feel it is easier and more inspiring to learn about techniques in the information retrieval field when we can use the theories in a practical sense. Twitter was chosen as a domain because social media and micro blogging are poplar research fields. We also wanted to explore their API and see if we could make something interesting and enhance the application using theories from the information retrieval field.

%faglig: lære mer om hvordan søkemotorer ranker ting, presentasjon. Lære med om verktøy som brukes. Er det et behov for å gjøre dette?
PageRank has seen great success as a ranking algorithm for general web search, popularised trough Google's own search engine. The scientific motivation behind FeedJam is to see if the page rank algorithm is feasible and applicable to a social network, and if the ranking algorithm and a new way of presenting search results will make it easier to get a quick overview of what tweets in the feed are important.