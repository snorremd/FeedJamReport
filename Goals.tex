When we landed on the FeedJam project, i.e. to make a new website for Twitter searches, we defined two primary goals and some subsidiary goals. Our first primary goals involved investigating wether the Google PageRank algorithm could be used on Twitter users and implementing an algorithm that could do this. Our second primary goal was to make a novel layout that would make it easier to gain an oversight of and read large collections of tweets. In addition to these primary goals we defined, over time, several subsidiary goals based on personal motivations and requirements that arose during the project life cycle. These goals were:

\begin{description}
  \item[Twitter API:] To connect to and use the Twitter API to acquire search results as well as information on users and trending topics.
  \item[Spring:] To implement a web application with the Spring framework.
  \item[AJAX in Spring:] Explore the possibility of using AJAX techniques with the Spring framework.
  \item[Caching:] Implement caching of applicable items received from the Twitter API.
  \item[User evaluation:] To evaluate the resulting web application.
\end{description}

\subsection{Motivation} %lisa
%personlig: sociale media er i winden. det vil være inteeressant å  gjøre noe med ir for sosiale medier.
Our personal motivation for making FeedJam was that we got the chance to acquire new programming skills. We feel it is easier and more inspiring to learn about techniques in the information retrieval field when we can use the theories in a practical sense. Twitter was chosen as a domain because social media and micro blogging are poplar research fields. We also wanted to explore their API and see if we could make something interesting and enhance the application using theories from the information retrieval field.

%faglig: lære mer om hvordan søkemotorer ranker ting, presentasjon. Lære med om verktøy som brukes. Er det et behov for å gjøre dette?
PageRank has seen great success as a ranking algorithm for general web search, popularised trough Google's own search engine. The scientific motivation behind FeedJam is to see if the page rank algorithm is feasible and applicable to a social network, and if the ranking algorithm and a new way of presenting search results will make it easier to get a quick overview of the important tweets in the feed.