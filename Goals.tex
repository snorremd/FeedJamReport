The goal of this project is to learn more about how ranking of results and the presentation of them effects the a user's perception of the search engine. The domain of our project will be Twitter. We will try to make an application that presents Tweets in way that makes it easier for users to see what Tweets are interesting/important and which are not. In order to achieve this we will experiment with Googles PageRank and see if it is applicable to Twitter. Another sub goal will be to experiment with a user interface that make it easier to follow the Tweet feed. To evaluate these goals we will conduct a small user evaluation of the application.

A secondary will be to explore different ways to retrieve information from Twitter and implement caching of the information.


%main goal: finne en ny måte å presentere twitter
%sub goal: sjekke om pagerank lar seg oversett til twitter
%sub goal: finne nytt grensesnitt for twitter som gjøre det lettere å følge twitter stømmen

%{\bf Explore a novel presentation of information}\newline



%{\bf Implement a ranking algorithm}\newline %
%In order for users to easily see what Tweets are important or interesting we wish to implement a ranking algorithm. Even though the perception of what's important is subjective a general opinion is an indication of what's important. We want to implement the Google PageRank algorithm on Twitter users.

%Secondary goals: caching, user evaluation

%{\bf Programming}\newline
%spring
%ajax i spring

