Semantic analysis of tweets

\subsection{Personalised searches}
Another potential future research area would be to capture the user's own preferences and interests and use these to customise the search output. For instance we could take into accout the user's location and the previous searches and the returned tweets' semantic data.

Today the major search engines, such as Google, personalise searches using metrics such as the user's browser, operating system, location, browser settings such as language and data about previous searches (Google Blog, 2012).

Pariser, Eli: beware online “filter bubbles” | video on ted.com. Available at: %http://www.ted.com/talks/eli_pariser_beware_online_filter_bubbles.html [Accessed March 7, 2012].

Google Blog (04.12.2004), Personalized Search for everyone, available from <http://googleblog.blogspot.com/2009/12/personalized-search-for-everyone.html> Downloaded 31.01.2012

Clustering of search results and/or the cached collection