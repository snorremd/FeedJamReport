\subsection{Semantic analysis of tweets} %Lisa
\label{semtanticAnalysation}

\subsection{Mobile Application}
Another point of future development/research would be to make the FeedJam webapplication into a mobile application for modern mobile operating systems such as iOS, Android and Windows Phone 7. There are several different ways of achieving this. 

One way would be to rewrite/cross-compile our current Spring application and rewrite the views in order to create a fully native mobile application for each operating system. There are however several difficulties with choosing this path. For instance the pagerank algorithm would probably be too taxing for the limited hardware of these devices. In addition we would be unable to distribute the collection of users, resulting in a severely limited application due to the Twitter API limitations outlined in \ref{twitterProblem}.

Another way of creating a mobile application would in essence be to rewrite the views. Through wrapping the current Javascript/jQuery logic and a new design into for instance Apache Cordova, a tool for wrapping HTML5/Javascript/CSS applications in a native web view on smart phones/tabets \cite{ApacheCordova}, we would be able to use the same controllers and models as the current web application. While this approach would require more of our web server we would be able to serve a sentralised database of UserRanks in addition to having a centralised database of Twitter users which in turn would reduce the number of requests required to do one search in the Twitter database.

\subsection{Clustering of search results and/or the cached collection} %snorre

\subsection{Permalinks for accessibility}
One major problem with FeedJam in its current iteration is the lack of permalinks. Permalinks are, as the name implies, permanent urls to specific content pages. Since we use Javascript the generation of such permalinks is not automatic due to the fact that we dynamically insert content into an already generated page. While it would not necessarily be hard to implement a permalink structure we would have to rewrite portions of both our database (in order to store tweets so that they are accessible at a later date) in addition to rewriting a large portion of the already existing Javascript.