\subsection{Semantic analysis of tweets} %Lisa

\subsection{Personalised searches} %Torstein
Another potential future research area would be to capture the user's own preferences and interests and use these to customise the search output. For instance we could take into accout the user's location and the previous searches and the returned tweets' semantic data.

Today the major search engines, such as Google, personalise searches using metrics such as the user's browser, operating system, location, browser settings such as language and data about previous searches (Google Blog, 2012).

Pariser, Eli: beware online “filter bubbles” | video on ted.com. Available at: %http://www.ted.com/talks/eli_pariser_beware_online_filter_bubbles.html [Accessed March 7, 2012].

Google Blog (04.12.2004), Personalized Search for everyone, available from <http://googleblog.blogspot.com/2009/12/personalized-search-for-everyone.html> Downloaded 31.01.2012

\subsection{Clustering of search results and/or the cached collection} %snorre

\subsection{Permalinks for accessibility}
One major problem with FeedJam in its current iteration is the lack of permalinks. Permalinks are, as the name implies, permanent urls to specific content pages. Since we use Javascript the generation of such permalinks is not automatic due to the fact that we dynamically insert content into an already generated page. While it would not necessarily be hard to implement a permalink structure we would have to rewrite portions of both our database (in order to store tweets so that they are accessible at a later date) in addition to rewriting a large portion of the already existing Javascript.