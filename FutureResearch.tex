\subsection{Semantic analysis of tweets} %Lisa
\label{semtanticAnalysation}
To further improve the ranking in FeedJam it would be possible to include elements of semantic technology. We could collect data from other social networks like LinkedIn and Facebook, add semantics to the data using linked data and natural language processing (NLP) and use the data to populate an ontology user model. With this model it can be possible to infer if the user tweets about something that he or she knows a lot about or just posting something random. This could be used to give extra weight in the ranking function of FeedJam. 

Here is an example using LinkedIn: Through LinkedIn it is possible to retrieve information about the persons profession, their experience and skills. Through adding semantics to the data by placing them in an ontology, it is possible to infer which skills the user has about the topic. Hence it is possible to infer if he or she is a "good" tweeter about the topic in question.

\subsection{Mobile Application}
Another point of future development/research would be to make the FeedJam web application into a mobile application for modern mobile operating systems such as iOS, Android and Windows Phone 7. There are several different ways of achieving this. 

One way would be to rewrite/cross-compile our current Spring application and rewrite the views in order to create a fully native mobile application for each operating system. There are however several difficulties with choosing this path. For instance the page rank algorithm would probably be too taxing for the limited hardware of these devices. In addition we would be unable to distribute the collection of users, resulting in a severely limited application due to the Twitter API limitations outlined in \ref{twitterProblem}.

Another way of creating a mobile application would in essence be to rewrite the views. Through wrapping the current Javascript/jQuery logic and a new design into for instance Apache Cordova, a tool for wrapping HTML5/Javascript/CSS applications in a native web view on smart phones/tabets \cite{ApacheCordova}, we would be able to use the same controllers and models as the current web application. While this approach would require more of our web server we would be able to serve a centralised database of UserRanks in addition to having a centralised database of Twitter users which in turn would reduce the number of requests required to do one search in the Twitter database.

\subsection{Clustering of search results and/or the cached collection} %snorre
\citet[pg. 286]{Baeza-Yates2011} defines text clustering as, ``\textit{[\dots]given a collection D of documents, a text clustering method automatically separates these documents into K clusters according to some predefined criteria}''. Clustering algorithms are one type of text classification algorithms which can be divided into two main categories, unsupervised and supervised text classification algorithms. Clustering algorithms falls under the unsupervised category as they do not use any artificial intelligence techniques to classify the text. The predefined criteria varies between different clustering algorithm, and some clustering algorithms produce distinct clusters while others produce overlapping clusters. Generally speaking, a clustering algorithm will split a document collection into groups of documents with matching text content.

Several clustering algorithms exist such as K-means clustering, hierarchical agglomerative clustering, suffix tree clustering, etc. \cite{Baeza-Yates2011} Clustering is not only applicable to collections of large text documents, but could also be used to cluster Tweets or even users. A possible research problem could be to investigate how we could use clustering techniques to cluster search results in FeedJam. A possible scenario here would be to allow searches with for example 100 search results. These results could then be clustered and presented to the user in a visual cluster with some sort of label.

Another possible research problem worth investigating is how one could use a clustering technique to enhance or help the ranking algorithm. One of the main problems with the ranking algorithm and ranking Twitter users is the very large size of the network. One possible solution to this problem is to make domain or subject specific user ranks. Clustering algorithms could help identify such domains or subjects by clustering the Tweets into groups and then assigning users to these groups.

\subsection{Permalinks for accessibility}
One major problem with FeedJam in its current iteration is the lack of permalinks. Permalinks are, as the name implies, permanent urls to specific content pages. Since we use Javascript the generation of such permalinks is not automatic due to the fact that we dynamically insert content into an already generated page. While it would not necessarily be hard to implement a permalink structure we would have to rewrite portions of both our database (in order to store tweets so that they are accessible at a later date) in addition to rewriting a large portion of the already existing Javascript.