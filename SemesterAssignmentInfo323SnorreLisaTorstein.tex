\listfiles
\documentclass[a4paper,pt12]{article}
\usepackage[utf8x]{inputenc}

\usepackage[english]{babel}
\usepackage[pdftex]{graphicx}
\usepackage{amsmath}
\usepackage{pdfpages}
\usepackage{fancyhdr}
\usepackage{anysize}
\usepackage{subfig}
\usepackage{graphicx}
\marginsize{2,5 cm}{2,5 cm}{2,5 cm}{2,5 cm}
\usepackage[numbers]{natbib}
\bibpunct{(}{)}{;}{a}{,}{,}
\usepackage[colorlinks=true,allcolors=black]{hyperref}
\usepackage{booktabs}
\usepackage{numprint}
\usepackage[parfill]{parskip} % linje i stedet for indent i ny paragraf
\usepackage{setspace}
\onehalfspacing

%\usepackage{harvard}
\newcommand{\HRule}{\rule{\linewidth}{0.5mm}}
\begin{document}

\begin{titlepage}

\begin{center}


% Upper part of the page
\includegraphics[width=0.5\textwidth]{figures/feedjam.png}\\[1cm]    

\textsc{\LARGE University of Bergen}\\[1.5cm]

\textsc{\Large INFO323 Semester Project}\\[0.5cm]


% Title
\HRule \\[0.4cm]
{ \huge \bfseries FeedJam: a Twitter PageRank experiment}\\[0.4cm]

\HRule \\[1.5cm]

% Author and supervisor
\begin{minipage}{1\textwidth}
\begin{center} \large
\emph{Authors:}\\
Torstein \textsc{Thune}, Lisa \textsc{Halvorsen}, Snorre \textsc{Davøen}
\end{center}
\end{minipage}


\vfill

% Bottom of the page
{\large \today}

\end{center}

\end{titlepage}

\begin{abstract}
FeedJam is an experiment where we try to combine the PageRank algorithm and a novel presentation format in order to ease digestion of large conversations on Twitter. Through employing the Kanban development methodology and the Spring framework we created a fully working prototype.

The aim of the project is to implement a version of the PageRank rating algorithm to Twitter tweets, and present the result in a novel way. The software development method we will be using to develop the project is Kanban. We will develop a high fidelity working prototype using the Spring framework. It will retrieve tweets based on the user’s search query, rank the results using the PageRank algorithm and display the results in a website.     

\end{abstract}
\newpage

\tableofcontents
\newpage

\section{Introduction}
Twitter is a website where people can share short text messages about everything from simple personal doings to live reports from emerging world news stories (Twitter, 2012). Although much important information is often passed through Twitter, both as original tweets and retweets, it is often hard to follow the course of the discussion or information sharing when subscribing to certain topics due to the sheer number of tweets. Now, imagine if you could go to a specialised search engine, type in a topic or a user and then get an overview of the relevant tweets rated by importance.

Our project “FeedJam” utilises the power of the Java MVC; Model, View and Controller; Spring framework, jQuery, HTML 5, CSS3, JavaScript AJAX techniques and the Twitter REST API and our own ranking algorithm in order to search, retrieve, rank and display tweets and their respective users. In order to alleviate the effect of limitations in the Twitter API we use caching techniques, namely caching results in a MySQL database. The application can be deployed to both Jetty and Apache Tomcat web servers running on any Unix-based or Windows NT-based systems.

In this report we will present the initial goals of the project in greater detail (section~\ref{sec:goals}), with a focus on the PageRank algorithm and a novel presentation format. The report will also go into some detail about the underlying theory of ranking in web retrieval and presentation of search results (section~\ref{sec:background}). The application will also be fully detailed in the report, providing information about architecture, implementation of an standard PageRank algorithm in a new setting, and a novel layout for presenting search results (section~\ref{sec:application}). A small team and complex functionality require a agile development methodology and the right software technology. Section~\ref{sec:development} will give an overview of these subjects. While developing FeedJam we ran into several problems. One of them was the practical implementation of the user rank algorithm. Because these problems were a major difficulty in the development process we dedicated a own section to them (section~\ref{sec:problems}). Last but not least, our report will summarise our level of reached goals (section~\ref{sec:levelOfReachedGoals}) and go into possible and/or interesting feature research areas that could benefit or be related to our system (section~\ref{sec:futureResearch}).


\section{Goals}
The goal of this project is to learn more about how ranking of results and the presentation of them effects the a user's perception of the search engine. The domain of our project will be Twitter. We will try to make an application that presents Tweets in way that makes it easier for users to see what Tweets are interesting/important and which are not. 


%main goal: finne en ny måte å presentere twitter
%sub goal: sjekke om pagerank lar seg oversett til twitter
%sub goal: finne nytt grensesnitt for twitter som gjøre det lettere å følge twitter stømmen

{\bf Explore a novel presentation of information}\newline



{\bf Implement a ranking algorithm}\newline %
In order for users to easily see what Tweets are important or interesting we wish to implement a ranking algorithm. Even though the perception of what's important is subjective a general opinion is an indication of what's important. We want to implement the Google PageRank algorithm on Twitter users.

Secondary goals: caching, user evaluation

{\bf Programming}\newline
%spring
%ajax i spring



\section{Background}
%teoridel
In this section we will present some background information about the topics that the project consist of. We'll star by introducing search engines and more specifically how page rank can be used to make a search engine better. After this we will give a short introduction to Twitter, the social media that is the domain of the project. The section will finish with a presentation of some related work and a subsection of your motivation for doing the project. The next section describes the methods that we used.


\subsection{Search Engines} 
\begin{quote}Information retrieval deals with the representation, storage, organization of, and access to information items \citep[p.1]{Baeza-Yates2011}.\end{quote} It should be done in such a way that users easily can access the information they are looking for. There are many factors contributing to a information retrieval system. Our application is a kind of search engine for Twitter. It has two focus points related to the information retrieval process. The first is ranking of the search results and the second is presenting the results.  %feedjam er ikke egentlig en søkemotor, men vi rangere og presentere resultater fra twitter søkemotoren


{\bf Ranking}
Ranking of results is an integral part of any search engine, no less in ours. But how exactly does one rank search results? “A second critical challenge is the identification of quality content in the Web. Evidence of quality can be indicated by several signals such as domain names [...], text content, and various counts (such as the number of word occurrences), links (like PageRank), and Web page access patterns as monitored by the search engine” \citep[p.468]{Baeza-Yates2011}
\citet{Baeza-Yates2011} go on to describe different types of signals that can be used to improve ranking of web results. These will be briefly described here. Content signals refers to the content of the document, mainly text. The signal in the content type refers to things such as word counts, vector space model-based scores and so forth. Structural signals, according to \citet{Baeza-Yates2011}, includes signals such as the links on a page, the text in anchor elements, and in and out-links to a page. The last signal type describes is the web usage signals. User clicks is one often used signal where the relevance of a result is affected by the number of clicks it receives. Other signals mentioned are geographical location, language, technological context and temporal context (that is web usage over time gathered through cookies or user profiles)\citep{Baeza-Yates2011}.
%signals kan være ting som vi trekker inn i tweet ranken (content based signals). Hashtags, firefox plugin greie--->skriv om det.

In FeedJam we use content based signals in the TweetRank. Mentions indicate some kind of communication between users. %maybe this can be an indication of the friend user. Maybe we dont want to emphasis these tweets? 
HashTags...
Retweets...
 See more about the implementation in section \ref{} \nameref{}

Link-based ranking is based on using structural signals (i.e. counting the number of outgoing and incoming links on a page) to rank a page or document. One ranking algorithm based on this  principle is the Google PageRank algorithm \citep{Page1999}.

\subsubsection{Page Rank}
The goal of the PageRank ranking algorithm is stated as follows: “This ranking called PageRank helps search engines and users quickly make sense of the vast heterogeneity of the World Wide Web” \citep[p. 1]{Page1999}. The PageRank algorithm exploits the structure of the web by looking at the number of backlinks and out links a web page has. The World Wide Web can thus be viewed as a graph with directed edges (backlinks and forward or outgoing links) between nodes (web pages). \citet{Page1999} note that giving high ranks just by looking at the number of backlinks (incoming links) is problematic because a site might get many backlinks from various web pages that are not very important. They provide a solution to this by proposing that not only the number of backlinks, but also the rank of the web pages for these backlinks can be used when ranking a web page. Given this solution, as shown by \citet{Page1999}, a page receives a high page rank if the sum of the sum of the page ranks of the backlinks is high.

%Page rank formler. vise enkel utregning... 
%Litt mer om hvordan pagerank kan brukes på Twitterusers. Lage en graph/tegning. Vise at det i teorien skal oversettes - argumentere. Sosiale nettver er ganske like weben. 

\subsubsection{User Interface}

{\bf Interaction design principles and user-centered design}\newline
In addition to exploring ranking algorithms we are going to develop a novel interface to display tweets and their rankings. In the book Interaction Design: Beyond Human-Computer Interaction Sharp et al (2007) present several factors who are paramount to good interaction design. In order to design good interaction one should take into account the usability principles as presented in the book. These are visibility (whether the functions/input mechanisms are visible or hidden), feedback (the interface should show that the system has received and used the input given), constraints (avoiding facilitating invalid user input for given contexts), consistency (similar operations use the same kinds of elements) and affordance (whether users can use familiarisation from experience in order to naturally understand what the interface element does). In addition to this the book also provides a set of metrics which can be utilised to measure the usability of an artefact.\newline

{\bf \noindent Classic search engine result pages}\newline
\citet[p.480]{Baeza-Yates2011} argues that existing search engines are heavily influenced by what they call a extreme simplicity rule. Basically this rule states that search engine user experience must be simplistic enough in nature that users understand it without any previous knowledge, or else they will simply switch to another search engine. This has lead to a very prominent interface paradigm where all the most prominent search engines use the classic search engine result page (SERP) \citep{Baeza-Yates2011}. 
As search engines have started to search several different content types the need to display several different media types in one unified result page have also emerged. This has resulted in two innovations: the “onebox” \citep{Baeza-Yates2011}, which displays specific information from one knowledge domain (e.g weather reports) and universal search results where one combines several different searches into one search result page.


\subsection{Twitter}
%litt om hvordan twitter kan være vanskelig å få oversikt over en strøm. finn statistikk om megden tweets i døgnet.
%vi kan tilby noe mer enn twitter search enginen gir. Vi tilbyr en mer oversiktlig tjeneste. 
%se om det finnes noe som sier noe om hvordna brukere bruekr twitter. 
When deciding what information retrieval (IR) techniques we could apply to Twitter we found that when searching for Tweets it not easy to see what tweets are important and which are not. If you search for at certain topic on Twitter you get at feed of all tweets that match the search query. It can be some users personal diary, a conversation between users, information sharing or spam. Following we wanted to use IR techniques to make it easier to see what Tweets are important. 

Twitter is a microblogging service available for more than 20 languages \citep{Twitter2012}. User post messages, called tweets, about what they find interesting which are read by other users. As Twitter has become popular it has received a broad user group. The difference between Twitter and other social networks is the way  that it defines the relationships in the network. On Twitters a user is following other user and is being followed by other users. The difference is that the relationship doesn't have to be reciprocal. Being a follower means you receive all the tweets of the users you follow. \citep{Kwak2010} finds that 77.9\% of the followers relationships on Twitter are non reciprocal. This is one of the features we will take advantage of in order to rank tweets. 

%Dette var noe jeg fant i en artikkel. OG det er ikke så vanskelig å implementer noe som sier om en bruker er venn, deler eller søker. Kanskje vi skal gjøre det?
Users use Twitter in different ways. \citet{Akshay2007} finds three man user categories: Information source, Friend and Information Seeker. Information sources are user that updates their status with regular intervals and has many followers. The number of followers is due to the value of the information they give through the Tweets. The Information Seeker is a user who rarely post Tweets, but follows other users on a regular basis. A Friend is a users with many reciprocal relationships. We may use these user categories in the UserRank. 

Twitter also has other features that can be used in ranking. Important words in the text of a tweet can be marked with a hash tags ("\#" ). Tweets can be retweeted which is a mechanism that empowers the spreading of the tweet. \citep{Kwak2010} finds that a retweeted tweet reaches an average of 1000 users no matter what the number of followers the author of the original tweet has. 

We used the Twitter REST API to retrieve tweets and user information. More on how we used it can be found in section \ref{} \nameref{} 

\subsection{Technology}
This section gives a short presentation of the techonologies we used in the project. 

\subsubsection{Javascript/jQuery} %torstein
\subsubsection{Media Queries} %torstein
\subsubsection{AJAX}%torstein
\subsubsection{Spring} %lisa
To develop the project we used the Spring Framework. It is a Java based open source application framework. We chose this framework because it has many modules that easily takes care of the "plumbing" of the application and makes our work easier \citep{SpringSourcec}. Some of the group members has used it before, and all wanted to get to know it better because it one of the big enterprise frameworks. The application is set up using Spring MVC architecture which allows for easy communication between the web view and the controller (more on this in section \ref{} \nameref{}) \citep{SpringSourcee}. FeedJam uses Spring's REST module for client-side HTTP access to Twitter's API. We also make user of Spring's data access module for connection to the database.

\subsubsection{Jetty/Tomcat} %lisa Jeg ver ikke helt hva jeg skal skrive her
In the development we have used Jetty WebServer and Apache Tomcat to test run the application. Jetty is a open source HTTP server, HTTP client and Java servlet container \citep{MortBayConsultinga}. 

\subsubsection{MySQL} %lisa Jeg vet ikke helt hva jeg skal skrive her
MySQL is the worlds most popular open source database \citep{OracleCorporationand/oritsaffiliates}.

\subsection{Related Work}
{\bf trst.me} \newline
trst.me \citet{Infochimps2012b} is tool for measuring a Twitter user's reputation. The measure is two dimensional and consists of the Trstrank and the Trstquitient. The Trstrank is a implementation of Google's page rank \citep{Infochimps2012a}, the same rank the FeedJam uses for the userRank. The Trstrank gives a score between 0 and 10 which gives many users the same rank. To differentiate between the users trst.me uses the Trstquotient. It is "an integer between 0 and 100 that quantifies the relationship between a user's Trstrank and their followers count"\citep{Infochimps2012}. The quotient is used to separate spammers from quality users. 

trst.me compared to FeedJam is just a service for finding a Twitter user's trustworthiness. It allows users to enter the screen name of a Twitter user and returns their rank along with other related metrics.  

\subsection{Motivation} %lisa
%personlig: sociale media er i winden. det vil være inteeressant å  gjøre noe med ir for sosiale medier.
The personal motivation for doing this assignment is that we got the chance to acquire new programming skills. We feel it is easier and more inspiring to learn about techniques in information retrieval field when we can use the theories in a practical sense. Twitter was chosen as a domain because social media and microbloging is a poplar research field. And we wanted to explore their API and see if we could make some interesting and enhance the application using theories from the information retrieval field. 

%faglig: lære mer om hvordan søkemotorer ranker ting, presentasjon. Lære med om verktøy som brukes. Er det et behov for å gjøre dette?
Social media are growing in popularity...



\section{The application}
%Short description of the posible actions in feedjam
FeedJam is a Twitter application that uses the Twitter REST API to retrieve data from Twitter. When entering the page the user can use the search field to search for their interests. The trending topics of the last 24 hours is also presented on the front page. Once the users has entered a search term the API's search method to retrieve tweets based on a query term. 
The system responds with relevant tweets from the last two weeks. It also retrieves information about the user that wrote the tweet. The application stores tweets, user information and trending topics in a database. 

%Sections
\subsection{Structure}

\subsubsection{MVC} %torstein view delen, lisa model og contoller
The application is built on Spring's Web MVC framework. MVC separates the business logic from the user interface through the controller that handles the communication between the two. In Spring the dispatcherServlet...%blabla


\subsubsection{The Model} %what it does. what is contains 
The model is the part of the application that contains the domain objects or the core data structures \citep{}%find martin fowler
It 
Factory
We have data access objects (DAO) 
\subsubsection{The View}
The view is the component of MVC-applications which receives input from the user and displays output from the controllers and model. While the model and the controllers contain logic in order to do database interactions and data manipulation the view in general only contains logic in the form of simple if-statements and loops in order to display the content assigned by the controller. The view is also the component which, in web applications, contains the HTML markup, CSS-styling and javascript needed to provide users with the intentioned graphical design and user experience.

Within the context of Spring views are written in a language called JSP. JSP, Java Server Pages, is an alternative form of Java intended to be used within HTML and then compiled server side. While it is possible to write entire applications in JSP it is generally not considered good practice to do any logic outside logic specifically needed in order to display content from within the objects assigned to the view.

\subsubsection{The controller}
The controller takes the user input from the view and sends it to the business logic that handles it and sends response back to the controller.
In Spring the Controller prepares a Map which is sent to the 

%hva MVC er
%hvor ting er
\subsubsection{Interaction between view and controller} %snorre

\subsubsection{Interaction between controller and model} %lisa
%information flow, hvordan ting skjer
FeedJam has tree controllers. The HomeController handles the trend requests for from the front page. It finds the time of the request and uses the MySQLTrendingFactory class to search the database for trends. If the trend exist in the database they are returned to the view. If trends for the current hour don't exist in the DB, a request is sent to Twitter. The response is cached and sent to the view. More on the database structure in section \ref{feedJamDatabase} \nameref{feedJamDatabase}.

AjajController handles all search requests. When the client has retrieved search results from Twitter, the Json response is posted to AjajController. Using the MySQLUserFactory, it queries the database to check if user information of the tweet owners is cached. If some users don't exist in the database, a response is sent to the view. It contains a Json array of users the client needs to retrieve information about. 
The AjajConntroller then receives a POST containing the the information on the remaining users information. New users and the Tweets are inserted to the database and a TweetSearchResult object is created. The result contains the requested Tweets with associated user information. The result is then sent as a response to the view. When the client has received the response from the server it requests the users followers and following. The list containing these users is sent to AjajController which uses MySQLUserFactory to inserts them to the database. 

The SearchController is now used as a back up. It originally handled server side requests to Twitter. %more on this? json ++

%brukt desingn patterns 
{\bf packages}
\begin{itemize}
  \item uib.info323.twitterAWSM contains the controllers
  \item uib.info323.twitterAWSM.exceptions contains exceptions
  \item uib.info323.twitterAWSM.io contains interfaces for interacting with persistence layer (postfixed -DAO), interfaces for searching the Twitter API (postfixed -SearchFactory) and interfaces for crating model object (postfixed -Factory).
  \item uib.info323.twitterAWSM.io.impl contains MySql implementations, Json implementations and model implementations. 
  \item uib.info323.twitterAWSM.io.rowmapper contains rowmappers used in accessing persistence layer
  \item uib.info323.twitterAWSM.model.impl contains model implementations
  \item uib.info323.twitterAWSM.model.interfaces contains model interfaces
  \item uib.info323.twitterAWSM.model.impl contains model implementations
  \item uib.info323.twitterAWSM.pagerank contains PageRank/UserRank implementation
  \item uib.info323.twitterAWSM.model.utils contains parsers etc.
\end{itemize}


\subsection{Layout} %torstein
A large part of our project was to develop a novel way of displaying tweets. This section will explain how we developed the layout concept in addition to explaining how we implemented it using modern standards and techniques.

\subsubsection{Defining the concept}
Since FeedJam is an effort to ease the consumption of tweets and Twitter conversations through the use of the Page Rank algorithm to rank users and a simple tweet ranking we needed a good way of displaying the results of said rankings graphically. Normally search engines are able to sort the output of searches after relevance due to their non-restricted access to their own database. However, in the case of Twitter we only have access to a severely limited number of tweets through their public API. In addition to this we believe that the shere amount of new tweets every minute would be a too large amount to effectively process for our computers and server even if we were able to fully access the Twitter database. Therefore a normal sorted listing of search results were rendered effectively impossible. Another aspect to take into consideration is the conversational nature of many tweets. It was decided that the FeedJam layout would somehow follow the conversation without sorting tweets but simultaniously helping users skim through large conversations quickly and help users see important tweets.

Normally when browsing through lists of unsorted rated content we are explicitly shown the content's rating. This is true for instance in the case of movie or music listings where a grade is displayed, often in the form of a dice or a number. While we did want to display our rating we also wanted to provide the user with more powerfull visual cues to the generated importance measure of tweets in order to follow our layout goal of easing skimming of large conversations. Hence we decided on the use of purely visual cues in the form of colours or transparency in order to differentiate between rankings.

\subsubsection{Early efforts}
Early concepts placed tweets in a list format. The thinking behind using a simple list format is that Twitter in its current iteration use a similar design to display tweets. This did however prove to be an inefficient use of the available space in the browser, and was quickly scrapped in favour of a grid-based layout.

\begin{figure}[ht]
    \begin{minipage}[b]{0.5\linewidth}
        \centering
        \includegraphics[width=\textwidth]{figures/twitter_list}
        \caption{Twitter in its current iteration.}
        \label{fig:Twitter}
    \end{minipage}
    \hspace{0.5cm}
    \begin{minipage}[b]{0.5\linewidth}
        \centering
        \includegraphics[width=\textwidth]{figures/layout_colour_borders}
        \caption{Early FeedJam concept}
        \label{fig:FeedJamColours}
    \end{minipage}
\end{figure}

We did also experiment with using colours to display the importance of tweets. Early thoughts used a traffic light metaphor where green signalised a good ranking, yellow a neutral ranking and red a bad ranking. The colours were used as border colours, as seen in figure  \ref{fig:FeedJamColours}. This did however prove to be a too insignificant visual cue, as in it took too long time for us to actually determine the rating of tweets through tis colour use, we also decided that the colours cluttered the design, making it less aestethical. Another problem was that we were only able to create three ranking brackets using our traffic light metaphor, which took away from the nuances of our ranking algorithm.

\begin{figure}[ht]
    \begin{minipage}[b]{1\linewidth}
        \centering
        \includegraphics[width=0.5\textwidth]{figures/layout_transparency}
        \caption{FeedJam concept using transparency}
        \label{fig:FeedJamTransparency}
    \end{minipage}
\end{figure}


Since our ranking algorithms has an output in the form of a float number (a number betweet 0.0 and 1.0) we decided that opacity, whose CSS operator can take in values in a float format, would be better since we would be able to display the nuances of our ranking algorithm to within two decimals.


\subsubsection{The final layout}
When visiting the FeedJam web application the users are presented with a clean layout consisting of a logo, a search box, a list of trending tweets and a footer containing some information about the project and its creators. The colour scheme is very simplistic, relying on red and different shades of gray, resulting in a minimalistic design.

\begin{figure}[ht]
    \begin{minipage}[b]{0.5\linewidth}
        \centering
        \includegraphics[width=\textwidth]{figures/feedjam_final_frontpage}
        \caption{FeedJam frontpage on laptop/desktop computers}
        \label{fig:FeedJamFrontpage}
    \end{minipage}
    \hspace{0.5cm}
    \begin{minipage}[b]{0.5\linewidth}
        \centering
        \includegraphics[width=\textwidth]{figures/feedjam_responsive_frontpage}
        \caption{FeedJam frontpage on tablets}
        \label{fig:FeedJamFrontpageTablet}
    \end{minipage}
\end{figure}

The search form consists of a standard, though large, search input field and a search button, thus providing both good visibility, since the form is very prominent, and affordance, since users are used to the concept of a search box and thus knows how to use it.

When entering a search term and clicking the search button (or when clicking a trending topic) the front page fades out, and a loading bar appears displaying information about what is being done. While the site exhibits unusual form behaviour, e.g not causing a page reload when clicking submit, this loading bar, with its connected loading messages, shows the user that something has happened and that something is still happening, thus fulfilling the feedback design principle.

\begin{figure}[ht]
    \begin{minipage}[b]{1\linewidth}
        \centering
        \includegraphics[width=0.5\textwidth]{figures/feedjam_loading}
        \caption{The loading bar}
        \label{fig:FeedJamLoading}
    \end{minipage}
\end{figure}


The search result page consists of a strict grid with 4 columns (or less depending on screen resolution) where each tweet is set at a specific width. Each tweet is in its own box with varying opacity depending on the relevant rankings. In order to more clearly show where tweets are in the grid even on lower opacities the border is always at 100\% opacity. While some tweets are almost transparent they will become fully visible when pointed at by the mouse pointer or clicked on tablets and smart phones. 


\subsubsection{Coding of the layout}
The frontend is coded using standard HTML5 for markup, CSS3 for styling and Javascript for front-end scripting. In addition we use JSPs, which are in essence compiled template files which are used to generate the views presented to the user.

The frontend is organised in several views. Roughly we can divide these views into two categories: main views and views which are included into other views. FeedJam has one main view called home.jsp. This view is generates the markup which the user is presented with when he/she first visits the feedjam website. The home view also uses the views footer.jsp which contains the markup for the footer (bottom of page), htmlheader.jsp which contains javascript declarations and jQuery initialisation and header.jsp which contains the search form. Other views is the tweetList.jsp view and the trendingList.jsp view which are used to generate grids of tweets and a slider of trending topics. These are placed dynamically within the home.jsp view through the use of Javascript.

In order to make the site function more dynamically we use a technique called AJAX (Asynchronous Javascript and XML) coupled with JSON and HTML. AJAX is a relatively new technique for dynamically running requests and using the returned data in some way using Javascript without causing page reloads. This also makes us able to do all the API calls to Twitter through each individual user's own browser, thus making the site able to serve up to 150 searches per user per hour as opposed to a total of 150 searches for the whole web application if we were to do these requests server-side. The data is then processed server side, and a view is returned to the client, which then inserts this dynamically into the already generated view.

In order to ease the AJAX calls in addition to ensuring that the Javascript works cross-browser we use jQuery Javascript library. This library provides easy-to-use methods for initiating both POST and GET requests with JSON data in addition to providing us with a good interface for DOM-manipulation (the DOM - Document Object Model - being the data model used for web sites.

FeedJam also relies on two third-party jQuery plugins. Revolver (TODO) is a jQuery plugin which provides slider functionality to any collection of HTML-elements. Masonry (todo) is a plugin which positions HTML elements properly in a grid, fixing some short-comings in the way different elements float in a standard HTML grid.



%Backend: Spring, Spring mvc, maven,  DAO, factorys, running on java servers, db more in db section,  developement
%\subsection{Backend} %Serverside?
%The application is developed in Spring MVC framework. 





%Ranking: implementation, reference
\subsection{Ranking} %lisa
\label{ranking}
When a user enters a search term FeedJam collects response from the Twitter search API. This API has some restrictions (read more on )%using search
The consequences of this is that FeedJam only can present new tweets. Because of the limitations on the information that is possible to receive from the Twitter API the FeedJam TweetRanking function is very simple. There are two dimensions to our TweetRank. If the Tweet is read by many, it is considered important. But because we measuring how many actually reads the Tweets would be another project, we use the retweets count as way to see how many potentially could red the Tweet. ... claim that a tweet that is retweeted has a potential to reach an awerage of 1000 users independently of how many follower the owner of the tweet has. Based on this we made a scale... % more on the scale

The other dimension consernes the markup in the Tweets. If the Tweet containing specific markup are more valuble, but if a Tweet contains to many markups it probably is a spam Tweet \citep{}.%Find spam ref 

The Mentions markup is used to annotate Twitter users in at Tweet. A Tweet that mentions another user will show on the other users profile timeline.
A Tweet receives zero points i

The results are ranked by the FeedJam Rank which is a combination of User Rank (based on Page Rank) and TweetRank. The UserRank uses the information about the tweeters followers and the users the tweeter follows. The relation between user/follower/following represents the link relation between pages in PageRank. The idea is that tweeters with many followers are interesting. They get many followers because they have something interesting to tweet about. 

%caching DB 
\subsection{Database} %lisa
\label{feedJamDatabase}
The application uses a MySQL database to store information from Twitter. This was nececary because the Twitter API has restrictions on number of request one can make to it.
We user the DAO pattern. The DAO pattern is used as an abstraction of the connection between the domain objects and datasource. The DAO interfaces TrendDAO, TweetDAO and UserDAO defines the operations that are alowed to do.  


%TweetRank
%A tweet is considered “good” if it’s read by many. 
%se på retweets of antall følgere de som har retweetet har.
%-hvis en tweet fra en bruker med lav userRank er retweetet av brukere med mange følgere/høy userRank blir tweetRank høy -> lav UserRank + høy TweetRank = middels FeedJamRank
%-hvis en tweet er tweetet and en bruker med høy UserRank men ikke retweetet -> høy UserRank + law TweetRank = middels FeedJamRank.
%- Høy userRank + Høy tweetRank = høy FeedJamRank
%HashTags: 0-1 = 1, 2-3 = 2; 4++ = 0
%Av de 100
%TweetRank = sum(retweeters follwers rank)
%log skala



\section{Development}
\subsection{Tools} %snorre
During the project life cycle a number of tools have been used for the development process, implementation of code and deployment of the web application itself. A brief description of the tools used will be provided in this section.

\subsubsection{Implementation}
To implement the Java and Spring Framework based backend the SpringSource Tool Suite was used. \textit{``SpringSource Tool Suite™ (STS) provides the best Eclipse-powered development environment for building Spring-powered enterprise applications.''} \cite{SpringSource}. SpringSource Tool Suite is built on the open-source IDE framework Eclipse, and provides functionality such as close integration with the Spring Framework, maven integration, automatic application deployment to web servers, debugging, code-completion and much more.

The front-end code was implemented using mainly Notepad++ for coding and Google Chrome development tools for debugging and testing. \textit{``Notepad++ is a free (as in "free speech" and also as in "free beer") source code editor and Notepad replacement that supports several languages. Running in the MS Windows environment, its use is governed by GPL License.''} \cite{Ho2012}.

\subsubsection{Deployment}
The application is deployed on a server running the Linux-distro Ubuntu 12.04, a Tomcat web server, a LAMP-stack (Linux, Apache, MySQL, PHP) and Maven. Ubuntu 12.04 is the newest version of the operative system developed by Canonical and is suitable for use as a web server with desktop access. To serve the webpages of the FeedJam application we use Tomcat. 

%server
\subsection{Workflow}
\subsection{Methodology}
%argumenter hvorfor for vi har valgt det.
%Tilpassinger, prsaker til tilpassinger.

This project has been developed using the Kanban-methodology which is a lean (and agile) development methodology


This project has been developed using a Kanban. Kanban is a Lean software development methodology with five core principles:
Vizualize work flow 
Limit work in progress 
Manage flow
Make process policies explicite
Improve collaboration

At the start of the project we made a backlog of User Stories. We used an electronic kanban board from Kanbanpad.com to visualize the work flow. Through the development we split the stories into smaller development tasks and posted on the kanban board. This helped us limit the work in progress because we would finish a task before starting a new one. (See attachments for screenshots from the kanban board). 

Technologies

Javascript/jQuery
HTML 5
CSS 3
Media Queries
AJAX
Git
Java
Spring
Jetty/Tomcat
MySQL
Workflow

Git/GitHub
Kanban

\section{Problems}
%Twitter api, out of requests
\subsection{Twitter API requests}

%Server probl
\subsection{Server Problems}

%fake data probl
\subsection{SQL/JSON injection}
Due to the Twitter API request limit and the resulting AJAX client workaround we have in essence opened our application to fraudulent input from the client. Since we have no control over the client, and due to the fact that the users have full control, they will be able to construct fake, but valid JSON data to send to our controllers. Basically this means that they will be able to create fake users, user data (such as descriptions), and fake the tables lising followers and following. They will also be able to dictate the content to show in the tweets we display.

There is no way that we can ensure that such fraudulent data does not make its way into our database outside of conducting all the queries server side, which as previously explained proved to be a bad solution due to the request limit. While it is possible to insert correctly formatted data, it is not possible to inject SQL statements due to our use of named parameters. In addition to this the fraudulent data problem is somewhat alleviated by our revisiting policy which ensures that this fraudulent data will be exchanged for new data after a period of 4 weeks.


\section{Level of reached goals}
As our live application located at http://feedjam.thunemedia.no shows we have been able to reach our main goals. Our application is able to retrieve data from the open Twitter APIs through the AJAX technique, and we are able to display the retrieved data in a novel interface. We have also been able to successfully implement a version of the PageRank algorithm which in addition to the simple TweetRank algorithm dictates the opacity of tweets. The whole application has been implemented in Spring, using a database to cache as much information as possible.

While most goals have clearly been reached in some respects there are several dimensions to reaching a goal. One major fault in our application is our implementation of the PageRank algorithm, which requires too much resources to be run effectively on our hardware. This, coupled with the limited amount of information we have been able to retrieve from Twitter so far, means that most users displayed in the live web application have not actually been ranked by the algorithm.

While we do cache userdata we do not at present cache queries. While caching queries and their responses would be quite useless in the application's current implementation it could be useful in order to provide data for defining the algorithms used to semantically analyse tweets. It could also be useful in cases where users want to share a particular conversation since we would be able to serve the conversation, as it was at a specific time, without having to reaccess data through the Twitter API.

Whether our concept works at its foundation can also be discussed. As the application works now popular users will be more visible than less popular users. This seems to work in cases where the topic discussed is frequented by the top tier of Twitter users. However, in topics where only lesser known users are active all tweets will become semi-transparent, often at approximately the same level of transparency, thus rendering our efforts useless. While this is true in the application in its current format, we believe that normalising the scores according to topic would solve this problem.

Due to problems with the implementation of the system, especially with regards to the implementation of PageRank, we have been unable to conduct a proper user evaluation of the system. We have however experienced that the users we have introduced to the system seem to have a generally positive attitude to the concept and the implemented interface.

\section{Evaluation}
\subsection{Evaluating the layout} %torstein
In order to evaluate the effectiveness of our layout we performed a simple usability test, though due to the fact that our layout is extremely simple and small in scope we did not perform an in depth evaluation.

TODO

\subsection{Effectiveness of the ranking algorithms}
In order to evaluate the real world effectiveness of our ranking algorithms we created a simple rating script where the users themselves were asked to rate tweets from 0 (completely unrelevant) to 10 (best possible fit). These user-provided ratings were then compared to our internal FeedJamRank.

Discussion
Comparison of results
Problems

\section{Furture Research}
\subsection{Semantic analysis of tweets} %Lisa

\subsection{Personalised searches} %Torstein
Another potential future research area would be to capture the user's own preferences and interests and use these to customise the search output. For instance we could take into accout the user's location and the previous searches and the returned tweets' semantic data.

Today the major search engines, such as Google, personalise searches using metrics such as the user's browser, operating system, location, browser settings such as language and data about previous searches (Google Blog, 2012).

Pariser, Eli: beware online “filter bubbles” | video on ted.com. Available at: %http://www.ted.com/talks/eli_pariser_beware_online_filter_bubbles.html [Accessed March 7, 2012].

Google Blog (04.12.2004), Personalized Search for everyone, available from <http://googleblog.blogspot.com/2009/12/personalized-search-for-everyone.html> Downloaded 31.01.2012

\subsection{Clustering of search results and/or the cached collection} %snorre

\subsection{Permalinks for accessibility}
One major problem with FeedJam in its current iteration is the lack of permalinks. Permalinks are, as the name implies, permanent urls to specific content pages. Since we use Javascript the generation of such permalinks is not automatic due to the fact that we dynamically insert content into an already generated page. While it would not necessarily be hard to implement a permalink structure we would have to rewrite portions of both our database (in order to store tweets so that they are accessible at a later date) in addition to rewriting a large portion of the already existing Javascript.

\section{Conclusion}
Through our work on the FeedJam application we experienced how severely limited the fucking Twitter API is.

managed to implement pagerank
- but limited server capacity
- Twitter too connected

managed to create a novel interface

had many problems related to the twitter api

no precision evaluation due to pagerank/server capaciy problems

LISA WORKED ON PROJECT => A

\input{Bibliography}
%\make a Bibliography.tex file and insert something like this
%\bibliographystyle{plain}
%\bibliography{references}

\end{document}
