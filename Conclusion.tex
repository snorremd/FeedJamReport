Through our work we have been able to implement a new novel way of displaying tweets, utilising information retrieval techniques and algorithms to differentiate between tweets. The application, which at the time of writing can be accessed at http://feedjam.thunemedia.no, was developed and deployed using modern frameworks, techniques and tools.

Although the project has in most respects been a success, resulting in an application which works, there are several components which does not work optimally. For instance the combination of old hardware and our implementation of the PageRank algorithm, in addition to Twitter users seemingly being more tightly connected than webpages, has led to us being unable to rank the more popular users in our database. In addition it can be discussed whether the concept itself in its current implementation actually fulfills our goal of easing the consumption of large collections of tweets.


managed to implement pagerank
- but limited server capacity
- Twitter too connected

managed to create a novel interface

had many problems related to the twitter api

no precision evaluation due to pagerank/server capaciy problems

LISA WORKED ON PROJECT => A