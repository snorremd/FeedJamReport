As our live application located at http://feedjam.thunemedia.no shows we have been able to reach our main goals. Our application is able to retrieve data from the open Twitter APIs through the AJAX technique, and we are able to display the retrieved data in a novel interface. We have also been able to successfully implement a version of the PageRank algorithm which in addition to the simple TweetRank algorithm dictates the opacity of tweets. The whole application has been implemented in Spring, using a database to cache as much information as possible.

While most goals have clearly been reached in some respects there are several dimensions to reaching a goal. One major fault in our application is our implementation of the PageRank algorithm, which requires too much resources to be run effectively on our hardware. This, coupled with the limited amount of information we have been able to retrieve from Twitter so far, means that most users displayed in the live web application have not actually been ranked by the algorithm.

While we do cache userdata we do not at present cache queries. While caching queries and their responses would be quite useless in the application's current implementation it could be useful in order to provide data for defining the algorithms used to semantically analyse tweets. It could also be useful in cases where users want to share a particular conversation since we would be able to serve the conversation, as it was at a specific time, without having to reaccess data through the Twitter API.

Whether our concept works at its foundation can also be discussed. As the application works now popular users will be more visible than less popular users. This seems to work in cases where the topic discussed is frequented by the top tier of Twitter users. However, in topics where only lesser known users are active all tweets will become semi-transparent, often at approximately the same level of transparency, thus rendering our efforts useless. While this is true in the application in its current format, we believe that normalising the scores according to topic would solve this problem.

Due to problems with the implementation of the system, especially with regards to the implementation of PageRank, we have been unable to conduct a proper user evaluation of the system. We have however experienced that the users we have introduced to the system seem to have a generally positive attitude to the concept and the implemented interface.