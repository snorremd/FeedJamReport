Twitter is a website where people can share short text messages about everything from simple personal doings to live reports from emerging world news stories (Twitter, 2012). Although much important information is often passed through Twitter, both as original tweets and retweets, it is often hard to follow the course of the discussion or information sharing when subscribing to certain topics due to the sheer number of tweets. Now, imagine if you could go to a specialised search engine, type in a topic or a user and then get an overview of the relevant Tweets rated by importance.

Our project “FeedJam” utilises the power of the Java MVC; Model, View and Controller; Spring framework, jQuery, HTML 5, CSS3, JavaScript AJAX techniques and the Twitter REST api and our own ranking algorithm in order to search, retrieve, rank and display tweets and their respective users. In order to alleviate the effect of limitations in the Twitter api we use caching techniques, namely caching results in a MySQL database. The application can be deployed to both Jetty and Apache Tomcat web servers running on any Unix-based or Windows NT-based systems.

In this report we will present the initial goals of the project in greater detail (section~\ref{sec:goals}), with a focus on the page rank algorithm and a novel presentation format. The report will also go into some detail about the underlying theory of ranking in web retrieval and presentation of search results (section~\ref{sec:background}). The application will also be fully detailed in the report, providing information about architecture, implementation of an standard page rank algorithm in a new setting, and a novel layout for presenting search results (section~\ref{sec:application}). A small team and complex functionality require a agile development methodology and the right software technology. Section~\ref{sec:development} will give an overview of these subjects. While developing FeedJam we ran into several problems. One of them was the practical implementation of the user rank algorithm. Because these problems were a major difficulty in the development process we dedicated a own section to them (section~\ref{sec:problems}). Last but not least, our report will summarise our level of reached goals (section~\ref{sec:levelOfReachedGoals}) and go into possible and/or interesting feature research areas that could benefit or be related to our system (section~\ref{sec:futureResearch}).