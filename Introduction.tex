Twitter is a website where people share short text messages about themes varying from simple personal doings to live reports from emerging world news stories (Twitter, 2012). Although much important information is often passed through Twitter, both as original Tweets and through retweets, it is often hard to follow the course of the discussion or information sharing when subscribing to certain topics due to the sheer number of Tweets.

Now, imagine if you could go to a specialised search engine, type in a topic or a user and then get an overview of the relevant Tweets rated by importance.

Our project “FeedJam” utilises the power of the Java MVC; Model, View and Controller; Spring framework, jQuery, HTML 5, CSS3 and JavaScript AJAX techniques and the Twitter REST api and our own ranking algorithm in order to search, retrieve, rank and display tweets and their respective users. In addition to this, in order to elevate the effect of limitations in the Twitter api we use caching techniques, namely caching results in a MySQL database. The application can be deployed to both Jetty and Apache Tomcat web servers running on any Unix-based or Windows NT-based systems.

In the next sections we will present the goals of the project in greater detail. Section (3) will present some background material.