%Short description of the posible actions in feedjam
In this section we will describe the finished FeedJam application.
%Sections
\subsection{Structure}

\subsubsection{MVC} %torstein view delen, lisa model og contoller
\label{sec:MVC}
The application is built on Spring's Web MVC framework. The MVC design pattern separates the business logic from the user interface through the controller that handles the communication between the two. The Spring MCV is request driven and designed around Springs Front Controller the DispatcherServlet that dispatches requests to controllers and offers other functionalities that facilitates development of web apps. The request goes through following steps:

\begin{itemize}
\item Client sends a request to the web container in the form of a http request
\item The request is handled by the DispatcherServlet which finds the appropriate Handler Mapping
\item The Handler Mapping helps the DispatcherServlet send the request to the right Controller. 
\item The controller processes the request and returns a view in the for of a ModelAndView instance that's sent to the DispatcherServlet. 
\item The DispatcherServlet resolves the view by consulting the View Resolver
\item And the view is then returned to the client. 
\end{itemize}

The structure of our web application is as follows:

\begin{itemize}
  \item uib.info323.twitterAWSM - contains the controllers
  \item uib.info323.twitterAWSM.exceptions - contains exceptions
  \item uib.info323.twitterAWSM.io - contains interfaces for interacting with persistence layer (postfixed -DAO), interfaces for searching the Twitter API (postfixed -SearchFactory) and interfaces for crating model object (postfixed -Factory).
  \item uib.info323.twitterAWSM.io.impl - contains MySql implementations, Json implementations and model implementations. 
  \item uib.info323.twitterAWSM.io.rowmapper - contains rowmappers used in accessing persistence layer
  \item uib.info323.twitterAWSM.model.impl - contains model implementations
  \item uib.info323.twitterAWSM.model.interfaces - contains model interfaces
  \item uib.info323.twitterAWSM.model.impl - contains model implementations
  \item uib.info323.twitterAWSM.pagerank - contains PageRank/UserRank implementation
  \item uib.info323.twitterAWSM.model.utils - contains parsers etc.
\end{itemize}

\subsubsection{The Model} %what it does. what is contains 
The model is the part of the application that contains the domain objects or the core data structures. In FeedJam the main interfaces are TwitterUserInfo323, TweetInfo323 and Trend. Trend represents the trending topics presented on the front page. While TwitterUserInfo323 represents the user including the information such as name, screen name, number of followers and following, tweets count etc. It also contain a field for fitnessScore which represents the user's UserRank. TweetInfo323 represents the tweets and info related to them. Some of the fields are text, retweetCount, tags, mentions and tweetRank. The tweet class also contains reference to the Twitter user that wrote the tweet (the field userInfo). Trends, TweetSearchResult and FollowersFollowingResultPage are interfaces used to create the object that are sent to the view. We use the DAO factory pattern to create the object models used in the application. This is explained in more detail in section \ref{sec:feedJamDatabase} \nameref{sec:feedJamDatabase}.

%caching DB 
\subsection{Database} %lisa
\label{sec:feedJamDatabase}
The application uses a MySQL database to store information on users and trending topics. This was necessary because the Twitter API has restrictions on the number of request one can make to it. Caching is used in search engines as a way to make the search engine be or seem faster \citep{Baeza-Yates2011}. We use it to leverage the Twitter request restrictions making the application faster. We cache trending topics and user info. We also have functionality for caching tweets that are not used at the moment, but can be used for functionality described in \nameref{sec:futureResearch} section \ref{sec:futureResearch}. An ER-diagram of the database can be found in figure \ref{fig:erDiagram}.

\begin{figure}[ht]
    \begin{minipage}[b]{1\linewidth}
        \centering
        \includegraphics[width=1\textwidth]{figures/erDiagram}
        \caption{An ER diagram of the FeedJam database}
        \label{fig:erDiagram}
    \end{minipage}
\end{figure}

%We use the DAO pattern to create an abstraction between the domain objects and the datasource.  The DAO interfaces TrendDAO, TweetDAO and UserDAO defines the operations that are alowed to do.  

The application use the Data Access Object (DAO) factory pattern to provide the interfaces for interacting with the database. FeedJam has three interfaces of this type: TweetDAO, UserDAO and TrendDAO. They define methods for different data access  operations. The corresponding MySQLUserFactory etc. classes implement the interfaces and provide functionality for selecting, updating and querying the MySQL database.

\subsubsection{The View}
The view is the component of MVC-applications which receives input from the user and displays output from the controllers and model. While the model and the controllers contain logic in order to do database interactions and data manipulation the view in general only contains logic in the form of simple if-statements and loops in order to display the content assigned by the controller. The view is also the component which, in web applications, contains the HTML markup, CSS-styling and Javascript needed to provide users with the intended graphical design and user experience.

Within the context of Spring, views are written in a language called JSP. JSP, Java Server Pages, is an alternative form of Java intended to be used to print variables within HTML. Unlike many other languages used for web programming like PHP, JSP pages are pre-compiled. While it is possible to write entire applications in JSP it is generally not considered good practice to write any logic outside logic specifically needed in order to display content from within the objects assigned to the view by the controller.

\subsubsection{The Controller}
%FeedJam has two controllers, HomeController and AjajController, one for each view in the application.
The controller is the part of the MVC-application which handles communication between the view and the business logic. The controller takes the user input from the view and sends it to the business logic which handles it and sends the response back to the controller which sends it to the view \citep{Oracle}. Our application has two primary controllers, the HomeController and the AjajController. 

The Ajaj controller receives requests from the view in JSON strings. The JSON strings are parsed by a JsonParser (utils-package) and the model produces appropriate responses. The controller prepares maps \footnote{java.util.Map \citep{Oraclea}} containing the view(s) and the response objects. A Spring InternalResoucesViewResolver\footnote{org.springframework.web.servlet.view.InternalResourceViewResolver \citep{TheSpring}}  generates views with information  from the maps. The Home controller acts in a similar fashion but receives regular HTTP Get requests which send input data in the URL.

%hva MVC er
%hvor ting er
\subsubsection{Interaction between view and controller} %snorre
\label{viewControllerInteraction}
When the user searches for a term on FeedJam a complex series of interactions happen between the view/the users browser and the FeedJam server. These interactions are illustrated in Figure~\ref{fig:AjaxInteractions}. When the user enters a search term into the FeedJam searchbox and clicks the search button the entered term is snapped up by Javascript. This search term is then, along with some other variables such as the number of tweets to fetch, formed into a GET request to the Twitter search API. The returned JSON is then sent to the AJAJ controller, which parses the JSON and generates a list of users whom are not in the database/or have not been updated for several weeks. This list of users is then returned to the client.

\begin{figure}[ht]
    \begin{minipage}[b]{1\linewidth}
        \centering
        \includegraphics[width=1\textwidth]{figures/sequencediagram}
        \caption{Sequence diagram for ajax interactions between browser, server and Twitter.}
        \label{fig:AjaxInteractions}
    \end{minipage}
\end{figure}

\begin{figure}[h!]
\begin{lstlisting}[language=javascript]
// sends following to controller (ajaj/processFollowing)
function followingToServer(userId, followingData) {
  $.post("ajaj/processFollowing", 
  { 
    userId: userId.toString(), // server wants userId as a string
    following: JSON.stringify(followingData) // and JSON as a string
  },
  function(data) {
    console.log("SERVER RESPONSE: added following for " + userId);
  });
}
\end{lstlisting}
\caption{Example jQuery code for sending following for a user as JSON to the controller}
\label{ajaxRequest}
\end{figure}

The client then requests user information about all these users from Twitter. The returned JSON-formatted user data is then forwarded to the controller along with the previously returned JSON-formatted tweets. The reason why we return the tweets a second time is due to the fact that we do not formulate a session for each search, thus rendering the controller unable to couple the two HTTP POST requests with eachother. The controller then processes the tweets and the returned users and finally returns a finished view as HTML to the client. This finished HTML is then dynamically appended to the tweet grid using Javascript.

After the generated view has been inserted the client starts requesting followers and following for each of the users who were looked up. This is done this lately in the process in order to ensure that the view is returned and displayed to the FeedJam user as soon as possible, thus making the system appear faster. Since we are using AJAX this can be done behind the scenes without the user noticing. Each response from the Twitter API is then asynchronously sent to the controller, which in turn processes the JSON and inserts the values into our database.

\subsubsection{Interaction between controller and model} %lisa
\label{modelControllerInteraction}
%information flow, hvordan ting skjer
FeedJam has three controllers. The HomeController handles the trend requests for the front page. It finds the time of the request and uses the MySQLTrendingFactory class to search the database for trends. If the trends exists in the database they are returned to the view. If trends for the current hour does not exist in the DB, a request is sent to the  Twitter API. The response is stored in the database and sent to the view. More on the database in section \ref{sec:feedJamDatabase} \nameref{sec:feedJamDatabase}.

The AjajController handles all search requests. When the client has retrieved search results from Twitter, the JSON response is posted to AjajController. Using the MySQLUserFactory, it queries the database to check if user information of the tweet owners is cached. If some users are not recorded in the database, a response is sent to the view. It contains a JSON array of users the client needs to retrieve information about. 

The AjajConntroller then receives a POST containing the the information on the remaining users information. New users and the Tweets are inserted to the database and a TweetSearchResult object is created. The result contains the requested Tweets with associated user information. The result is then sent as a response to the view. When the client has received the response from the server it requests the users' followers and following. The list containing these users is sent to AjajController which uses MySQLUserFactory to insert them into the database. The application logic for this operation is listed in Figure \ref{javaProcessFollowing}.

\begin{figure}[h!]
\begin{lstlisting}[language=java]
@RequestMapping(value = "/processFollowing", method = RequestMethod.POST)
public ResponseEntity<String> processFollowing(
  @RequestParam String userId,
  String following) 
  {
    long userIdLong = Long.parseLong(userId);
    
    // Parse the json string and store following in result page object
    FollowersFollowingResultPage followingResultPage = 
      JsonUserParser.jsonToFollowersFollowing(userIdLong, following);
    
    // Insert following objects into database
    int updated = 
    mySqlUserFactory.insertBatchFollowing(followingResultPage);

    // Return a HTTP OK status to the client				
    return new ResponseEntity<String>(HttpStatus.OK);
}
\end{lstlisting}
\caption{Example Java code for receiving and inserting following into the database.}
\label{javaProcessFollowing}
\end{figure}

The SearchController is now used as a backup. It originally handled server side requests to Twitter. As explained in section \ref{twitterProblem} \nameref{twitterProblem}, we had a large amount of problems connected with the limited amount of requests per ip-adress per hour. Since the SearchController performed API requests server-side this meant that FeedJam could serve a maximum of 150 searches per hour in total. While all API requests are handled client-side in the final FeedJam version we keep the SearchController as a backup in the cases where the users have disabled Javascript in their browser in order to provide a similar, though limited, experience for those users.

%brukt desingn patterns 


\subsection{Layout} %torstein
A large part of our project was to develop a novel way of displaying tweets. This section will explain how we developed the layout concept in addition to explaining how we implemented it using modern standards and techniques.

\subsubsection{Defining the concept}
Since FeedJam is an effort to ease the consumption of tweets and Twitter conversations through the use of the PageRank algorithm, to rank users, and a simple tweet ranking we needed a good way of displaying the results of said rankings graphically. Normally search engines are able to sort the output of searches after relevance due to their non-restricted access to their own database. However, in the case of Twitter, we only have access to a limited number of tweets through their public API. In addition to this we believe that the sheer amount of new tweets every minute would be a too large to effectively process for our equipment even if we were able to fully access the Twitter database. Therefore a normal sorted listing of search results were effectively rendered impossible. Another aspect to take into consideration is the conversational nature of many tweets. It was decided that the FeedJam layout would somehow follow the conversation without sorting tweets but simultaneously helping users skim through large conversations quickly and help users see important tweets.

Normally when browsing through lists of unsorted rated content we are explicitly shown the content's rating. This is true for instance in the case of movie or music listings where a grade is displayed, often in the form of a dice or a number. While we did want to display our rating we also wanted to provide the user with more powerful visual cues to the generated importance measure of tweets in order to follow our layout goal of easing skimming of large conversations. Hence we decided on the use of purely visual cues in the form of colours or transparency in order to differentiate between rankings.

\subsubsection{Early efforts}
Early concepts placed tweets in a list format. The thinking behind using a simple list format is that Twitter in its current iteration use a similar design to display tweets. This did however prove to be an inefficient use of the available space in the browser, and was quickly scrapped in favour of a grid-based layout.

\begin{figure}[ht]
    \begin{minipage}[b]{0.5\linewidth}
        \centering
        \includegraphics[width=\textwidth]{figures/twitter_list}
        \caption{Twitter in its current iteration.}
        \label{fig:Twitter}
    \end{minipage}
    \hspace{0.5cm}
    \begin{minipage}[b]{0.5\linewidth}
        \centering
        \includegraphics[width=\textwidth]{figures/layout_colour_borders}
        \caption{Early FeedJam concept}
        \label{fig:FeedJamColours}
    \end{minipage}
\end{figure}

We did also experiment with using colours to display the importance of tweets. Early thoughts used a traffic light metaphor where green signalised a good ranking, yellow a neutral ranking and red a bad ranking. The colours were used as border-colours, as seen in figure  \ref{fig:FeedJamColours}. This did however prove to be a too insignificant visual cue, leading to it taking too much cognition to perceive the actual ranking of tweets. While it could be argued that one with some training would be able to lessen the cognitive burden of this design, we decided that we wanted an even easier ranking layout concept. Another problem  with the traffic light metaphor was that we were only able to create three ranking brackets, which took away from the nuances of our ranking algorithm.

\begin{figure}[ht]
    \begin{minipage}[b]{1\linewidth}
        \centering
        \includegraphics[width=0.5\textwidth]{figures/layout_transparency}
        \caption{FeedJam concept using transparency}
        \label{fig:FeedJamTransparency}
    \end{minipage}
\end{figure}

Since our ranking algorithms has an output in the form of a float number (a number betweet 0.0 and 1.0) we decided that opacity, whose CSS operator can take in values in a float format, would be better suited. Using opacity we were also able to display the nuances of our ranking algorithm. We also found that this concept was less taxing cognitively, and that we were able to estimate the tweets' rankings without much cognitive thought. The best tweets, rendered completely visible, would also stand out compared with the other tweets to a much larger degree than when using the traffic light metaphor.

\subsubsection{The final layout}
When visiting the FeedJam web application the users are presented with a clean layout consisting of a logo, a search box, a list of trending tweets and a footer containing some information about the project and its creators. The colour scheme is very simplistic, relying on red and different shades of gray, resulting in a minimalistic design.

\begin{figure}[ht]
    \begin{minipage}[b]{0.5\linewidth}
        \centering
        \includegraphics[width=\textwidth]{figures/feedjam_final_frontpage}
        \caption{FeedJam frontpage on laptop/desktop computers}
        \label{fig:FeedJamFrontpage}
    \end{minipage}
    \hspace{0.5cm}
    \begin{minipage}[b]{0.5\linewidth}
        \centering
        \includegraphics[width=\textwidth]{figures/feedjam_responsive_frontpage}
        \caption{FeedJam frontpage on tablets}
        \label{fig:FeedJamFrontpageTablet}
    \end{minipage}
\end{figure}

The search form consists of a standard, though large, search input field and a search button, thus providing both good visibility, since the form is very prominent, and affordance, since users are used to the concept of a search box and thus knows how to use it.

When entering a search term and clicking the search button (or when clicking a trending topic) the front page fades out, and a loading bar appears displaying information about what is being done. While the site exhibits unusual form behaviour, e.g not causing a page reload when clicking submit, this loading bar; with its connected loading messages, shows the user that something has happened and that something is still happening, thus fulfilling the feedback design principle.

\begin{figure}[ht]
    \begin{minipage}[b]{1\linewidth}
        \centering
        \includegraphics[width=0.5\textwidth]{figures/feedjam_loading}
        \caption{The loading bar}
        \label{fig:FeedJamLoading}
    \end{minipage}
\end{figure}


The search result page consists of a strict grid with 4 columns (or less depending on screen resolution) where each tweet is set at a specific width. Each tweet is in its own box with varying opacity depending on the relevant rankings. In order to clearly show where tweets are in the grid even at lower opacities the border is always at 100\% opacity. While some tweets are almost transparent, they will become fully visible when pointed at by the mouse pointer or clicked on tablets and smart phones. This is done in order to make all tweets readable.

When clicking a username a small modal window appears displaying information about the user, including our user rank, the number of people following the user and the number of people followed by the user, the user's description from Twitter in addition to links to his/her web page and profile. This window can be closed either by re-clicking the user name or by clicking outside the modal window.

Beneath the grid of tweets there is a "load more" button. By clicking this button the user sends an AJAX request both to twitter to get the next 20 tweets which is then sent to the server. The server in turn goes through the same steps as when using the search box, finally resulting in a generated view (a tweet-grid) which is dynamically appended to the existing grid using jQuery.

\subsubsection{Coding of the layout}
The frontend is coded using standard HTML5 for markup/structure, CSS3 for styling and Javascript coupled with jQuery for front-end scripting. In addition we use JSPs, which are in essence compiled template files, to generate the HTML displayed to the user.

The frontend is organised in several views. Roughly we can divide these views into two categories: main views and views which are included into other views. FeedJam has one main view called home.jsp. This view generates the markup which the user is presented with when he/she first visits the feedjam website. The home view also uses the views footer.jsp which contains the markup for the footer (bottom of page), htmlheader.jsp which contains javascript declarations and jQuery initialisation and header.jsp which contains the search form. Other views is the tweetList.jsp view and the trendingList.jsp view which are used to generate grids of tweets and a slider of trending topics. These are placed dynamically within the home.jsp view through the use of jQuery.

%Backend: Spring, Spring mvc, maven,  DAO, factorys, running on java servers, db more in db section,  developement
%\subsection{Backend} %Serverside?
%The application is developed in Spring MVC framework. 

%Ranking: implementation, reference
\subsection{Ranking} %lisa
\label{ranking}
The FeedJam ranking consist of two parts, the TweetRank and the UserRank. It is displayed through the opacity of each tweet. The idea is that tweets tweeted by a user with high UserRank gets a high score. Tweets by high-ranking users will have close to 100\% opacity (no transparency) unless the tweet itself contains spam signals (see more below). The FeedJam ranking will also allow for normal users who follow good tweeting practice to get a good score.

\subsubsection{TweetRank}
The TweetRank is based on the idea of using ranking signals in search engines. Even though there isn't as many content signals in a tweet as in a normal web pages we use them to make the TweetRank. "\#" symbol is used in Twitter to mark keywords or topics in a tweet \citep{Twitter}. Twitter recommends using no more than 2 hash tags in a tweet. 
Mentions are another mark-up that is used in tweets. It indicates either that a tweet is a reply to another tweet or that a tweet concerns a user \citep{Twitterb}. If a tweet is read by many, it is considered important. The Twitter API does not provide this statistic. We have decided to use the retweet count of tweets as a heuristic to get an idea of how many times a tweet might have been read.

The TweetRank implementation can be found in TweetInfo323Impl class. It is calculated using the average score from tags, mentions and retweets. The points given are set by us and can be adjusted to give the desired effect. Because the opacity of the view is generated based on the tweetRank-score, both the TweetRank and the UserRank is included in this value. 

\subsubsection{UserRank}
The UserRank uses information about a user's followers and who a user follows to rank the user. The follower relation between to users represent the back-link (or inbound link) relationship in a page rank algorithm. The following relationship between two users represents the forward link (or outbound link) relationship. If you recall the three user categories described in section~\ref{Twitter}, it is clear that a user with a high user rank (i.e. high page rank) falls under the category Information source, and should therefore be regarded as an interesting user. It is probable that a highly ranked user have many followers, the followers themselves having a high user rank, because she makes interesting tweets.

The relationships between the users is similar to the we structure. Even if both distributions of pages vs. users are similar the problem when calculating UserRank is the enormous amounts of followers a user can have. According to twittercounter.com \citep{TwitterCounter}, Lady Gaga who is the most popular Twitter user, has over 25 000 000 followers. While in comparison Google which is ranked as the most popular web site by alexa.com \citet{AlexaInternet2012}, has about 5 000 000 inbound links. 

{\bf UserRank implementation}\newline
The UserRank implementation used in FeedJam is based on \citet{Goodrarzi2009}'s implementation of page ranking in Java. The first step when finding a user \emph{a}'s UserRank is to find all the followers of the user, then find their followers, etc. This is repeated until all related users are found and stored in a list (params). A matrix is then generated from the users in the params-list. Then a matrix containing a multifactor-constant is created. The multifactor is calculated based on the relations between the users, how many are following them and the damping factor (see source code UserRank.getMultiFactor). After this a matrix of 1- DAMPING\_FACTOR is made. These two matrixes makes out the linear equation that is the user's UserRank. The solving of the equation is left to a third party API \citep{Jama}. Lastly the rank of the user is picked out of the matrix.

 
%Litt mer om hvordan pagerank kan brukes på Twitterusers. Lage en graph/tegning. Vise at det i teorien skal oversettes - argumentere. Sosiale nettver er ganske like weben. 
{\bf Optimalisation} \newline
Our initial implementation of the ranking algorithm was extremely slow and ranking every user was simply not feasible due to the time requirement of the algorithm. To optimize the algorithm we performed a  simple time complexity analysis in order to identify critical operations in the ranking algorithm which could be optimised. To explain how we optimised the algorithm it is necessary to show some of the source code used in the ranking algorithm. Figure~\ref{fig:generateMatrix} shows the code that generates the matrix used in the algorithm. The generateMatrix method gets the multi factor for each cell in the matrix. The getMultiFactor method is run \(n^2\) times where n is the number of users in the parameter list.

\begin{figure}[h!]
\begin{lstlisting}[language=java]
private double[][] generateMatrix(){ 
  double[][] arr = new double[params.size()][params.size()]; 
    for(int i = 0; i < params.size(); i++){ 
      for(int j = 0; j < params.size(); j++){ 
        arr[i][j] = getMultFactor((String) params.get(i), (String) params.get(j));
      }
    } 
  return arr;
}
\end{lstlisting}
\caption{Java code for generating the linear equation parameter matrix \protect \cite{Goodrarzi2009}.}
\label{fig:generateMatrix}
\end{figure}

If one takes a close look at the original getMultiFactor method (Figure~\ref{fig:getmultifactor}) one can see that two database operations, getInbound and getOutBound, are performed for each matrix cell. This amounts to a time complexity of O(\( n^2\)) for these two database operations. If one presupposes that each database operation takes about 50 milliseconds to complete, then the time requirement of a user with a param list of 1000 users would be  \((1000^2 \cdot 50) \cdot 2\) milliseconds, which is about 1666 minutes. To optimize the algorithm we moved both database operations outside the double for-loop, thereby reducing the time-complexity of these operations to O(n) which is an significant improvement (Figure~\ref{fig:getmultifactorimproved}). Using the example above the time requirement would now be \(1000 \cdot 100\) which is 100 seconds. Calculating the multi factors is still a \(n^2\) problem, but every value is now in memory which should reduce the time used by the function. While this optimisation does consume more memory, the additional memory consumption is negligible compared to the vast reduction in computing time.

\begin{figure}[h!]
\begin{lstlisting}[language=java]
private double getMultFactor(String sourceId, String linkId){ 
  if(sourceId.equals(linkId)) 
    return 1;
  else{
    String[] inc = getInboundLinks(sourceId); 
    for(int i = 0; i < inc.length; i++){ 
      if(inc[i].equals(linkId)){ 
        return -1 *(DAMPING_FACTOR / getOutboundLinks(linkId).length);
      } 
    }
  } 
  return 0;
}
\end{lstlisting}
\caption{Java code for calculating the multi factors used in the linear equation parameter matrix \protect \cite{Goodrarzi2009}.}
\label{fig:getmultifactor}
\end{figure}

\begin{figure}[h!]
\begin{lstlisting}[language=java]
private double getMultiFactor(
  long sourceId, 
  long linkId, 
  Map<Long, Long[]> followers, 
  Map<Long, Integer> allNumberOfFollowing) 
  {
  if (sourceId == linkId) {
    return 1;
  } else {
    for (long follower : followers.get(sourceId)) {
      if (follower == linkId) {
        double factor = 0;
        try {
          long start = System.currentTimeMillis();
          factor = -1 * (DAMPING_FACTOR / allNumberOfFollowing.get(linkId));
        } catch (ArithmeticException ae) {
          return DEVIDED_BY_ZERO;
        }
        return factor;
      }
    }
  }
  return 0;
}
\end{lstlisting}
\caption{Improved getMultiFactor-method with reduced time-complexity for the database-operations.}
\label{fig:getmultifactorimproved}
\end{figure}

%\citep{boka kap11} caching is used in search engines a a means for making the search engine be or seem faster. we do caching of Twitter information. We desided to implement chariang of information from twitter because of the restriction on requests problem (see section \ref{} \nameref{}) 

-------- TODO --------

%We could also have used caching of query results. store the result of one query, but wouldnt have the oppdaterthet %
%one of the features of Twitter search. could have beenn used in some other function in the application like looking at the history of tweets. \citet{boka kap 11, under caching} sais that 50\% is cached


-------- TODO (kommer samme avsnitt to ganger her?) ---------



